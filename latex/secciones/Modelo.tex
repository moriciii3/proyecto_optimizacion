\subsection{Conjuntos}
Para resolver este problema de optimización, es necesario definir los conjuntos numéricos que faciliten su comprensión, modelación y resolución. A continuación, se presentan los conjuntos identificados:
\begin{itemize}
    \item $I = \{1,2,...,N\}$: Conjunto entero de trabajadores, siendo $N$ el número total de trabajadores y con $i\in I$.
    \item $D = \{1,2,...,H\}$: Conjunto entero de días, siendo $H$ el número total de días y con $d\in D$.
    \item $S = \{1,2,3\}$: Conjunto entero de turnos a trabajar, siendo 1 el turno de la mañana, 2 el de la tarde y 3 el de la noche, y con $s\in S$.
    \item $T = \{1,2,...,H/7\}$: Conjunto entero de semanas, con $t\in T$.
\end{itemize}
\subsection{Variables}
Una vez definidos los conjuntos necesarios, procede la definición de las variables que operarán sobre ellos, las cuales son fundamentales para la modelación y resolución del problema. Estas variables son las siguientes:
\begin{itemize}
    \item $X_{i,d,s} = 
    \begin{cases}
        1, & $Se asigna al trabajador $i$ en el día $d$ en la turno $s$.$\\
        0, & $etoc.$
    \end{cases}$
    \item $Y_{i,t} = 
    \begin{cases}
        1, & $El trabajador $i$ trabajó el fin de semana $t$.$\\
        0, & $etoc.$
    \end{cases}$
\end{itemize}
\subsection{Parámetros}
Para la construcción de la función objetivo y de las restricciones planteadas en el problema, es necesario identificar y definir los parámetros proporcionados en el enunciado, los cuales son fundamentales para su correcta resolución. Estos son:
\begin{itemize}
    \item $G_{i,d,s} = \{0,1,...,10\}$: La disposición del trabajador $i$ para el día $d$ en el turno $s$.
    \item $R_{d,s} = \{R \in \mathbb{N}\}$: La demanda de trabajadores el día $d$ en el turno $s$.
\end{itemize}
\subsection{Función Objetivo}
Una vez identificados los conjuntos, variables y parámetros, corresponde definir la función objetivo que operará sobre nuestro modelo matemático. Esta es:\\\\
máx $z=\displaystyle\sum_{d\in D} \sum_{i\in I} \sum_{s\in S}G_{i,d,s} \cdot X_{i,d,s}$\\\\
De forma resumida, la función objetivo busca maximizar la asignación de los trabajadores a sus turnos, utilizando la variable binaria $X_{i,d,s}$ y el parámetro $G_{i,d,s}$. De esta manera, se busca optimizar su desempeño y la calidad de atención brindada a los clientes.
\subsection{Restricciones}
Finalmente, nuestro modelo matemático requiere de un conjunto de restricciones que delimiten la zona factible de la función objetivo. Para ello, se identifican las siguientes restricciones a partir del enunciado, necesarias para obtener una solución optima del problema:
\begin{enumerate}
    \item No más de 2 turnos al día por trabajador:\\
    $\displaystyle\sum_{s=1}^{3}X_{i,d,s}\leq 2:\forall i\in I, \forall d\in D$
    \item Se prohíbe turno de noche luego turno de mañana del día siguiente por trabajador:\\
    $X_{i,d,3}+X_{i,d+1,1}\leq 1:\forall i\in I, \forall d\in \{1,2,..., H-1\}$
    \item Se debe cumplir con la demanda de trabajadores cada día:\\
    $\displaystyle\sum_{i\in I}X_{i,d,s}= R_{d,s}:\forall s\in S, \forall d\in D$
    \item No se puede trabajar 3 fin de semanas seguidos por trabajador:\\
    $X_{i,6\cdot t,s}+X_{i,7\cdot t,s}\ge Y_{i,t}: \forall i \in I, \forall s \in S, \forall t \in T$\\\\
    $Y_{i,t}+Y_{i,t+1}+Y_{i,t+2}\leq 2:\forall i\in I, \forall t\in \{1,2,...,(H/7)-1\}$
\end{enumerate}


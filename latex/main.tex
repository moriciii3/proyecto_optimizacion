\documentclass[a4paper,12pt]{article}
\usepackage[utf8]{inputenc}
\usepackage{amsmath, amssymb}
\usepackage{geometry}
\geometry{margin=2.5cm}

\title{Modelo de Asignación de Turnos en Clínica}
\author{}
\date{}

\begin{document}
	\maketitle
	
	\section*{Formulación Matemática del Modelo de Asignación de Turnos}
	
	\subsection*{Conjuntos e Índices}
	\begin{itemize}
		\item $i \in I = \{1, \dots, N\}$ : conjunto de trabajadores.
		\item $d \in D = \{1, \dots, H\}$ : conjunto de días en el horizonte de planificación.
		\item $s \in S = \{1, 2, 3\}$ : conjunto de turnos diarios, donde:
		\[
		1 = \text{mañana}, \quad 2 = \text{tarde}, \quad 3 = \text{noche}.
		\]
		\item $w \in W = \{1, \dots, \lfloor H/7 \rfloor\}$ : conjunto de semanas (para control de fines de semana).
	\end{itemize}
	
	\subsection*{Parámetros}
	\begin{itemize}
		\item $a_{i,d,s} \in [0,10]$ : nivel de disposición o compatibilidad del trabajador $i$ para trabajar el turno $s$ del día $d$ (0 = no disponible, 10 = totalmente dispuesto).
		\item $r_{d,s} \in \mathbb{Z}_{+}$ : demanda de trabajadores requerida en el turno $s$ del día $d$.
		\item $N$ : número total de trabajadores.
		\item $H$ : número total de días en el horizonte de planificación.
	\end{itemize}
	
	\subsection*{Variables de Decisión}
	\begin{itemize}
		\item $x_{i,d,s} =
		\begin{cases}
			1, & \text{si el trabajador $i$ es asignado al turno $s$ del día $d$,}\\
			0, & \text{en caso contrario.}
		\end{cases}$
		\item $y_{i,w} =
		\begin{cases}
			1, & \text{si el trabajador $i$ trabaja al menos un turno el fin de semana $w$,}\\
			0, & \text{en caso contrario.}
		\end{cases}$
	\end{itemize}
	
	\subsection*{Función Objetivo}
	\[
	\max \; Z = \sum_{i \in I} \sum_{d \in D} \sum_{s \in S} a_{i,d,s} \, x_{i,d,s}
	\]
	\noindent
	El objetivo es maximizar la suma total de las disposiciones de los trabajadores asignados, buscando la mayor satisfacción global posible sin comprometer la cobertura ni las condiciones operacionales.
	
	\subsection*{Restricciones}
	
	\paragraph{1. Límite de turnos diarios por trabajador}
	\[
	\sum_{s \in S} x_{i,d,s} \le 2 \quad \forall i \in I, \; \forall d \in D
	\]
	Ningún trabajador puede desempeñar más de dos turnos en un mismo día.
	
	\paragraph{2. Prohibición de turnos noche–mañana consecutivos}
	\[
	x_{i,d,3} + x_{i,d+1,1} \le 1 \quad \forall i \in I, \; \forall d \in D \setminus \{H\}
	\]
	Evita que un trabajador realice el turno de noche y al día siguiente el de mañana.
	
	\paragraph{3. Cobertura exacta de la demanda}
	\[
	\sum_{i \in I} x_{i,d,s} = r_{d,s} \quad \forall d \in D, \; \forall s \in S
	\]
	Cada turno debe estar cubierto exactamente por la cantidad de trabajadores requerida.
	
	\paragraph{4. Identificación de trabajo en fin de semana}
	\[
	y_{i,w} = 
	\begin{cases}
		1, & \text{si } \displaystyle \sum_{s \in S} (x_{i,7(w-1)+6,s} + x_{i,7(w-1)+7,s}) \ge 1, \\
		0, & \text{en otro caso.}
	\end{cases}
	\]
	Indica si un trabajador realiza al menos un turno durante sábado o domingo.
	
	\paragraph{5. Prohibición de tres fines de semana consecutivos trabajados}
	\[
	y_{i,w} + y_{i,w+1} + y_{i,w+2} \le 2 
	\quad \forall i \in I, \; \forall w \in W \setminus \{\lfloor H/7 \rfloor -2, \lfloor H/7 \rfloor -1, \lfloor H/7 \rfloor\}
	\]
	Evita que un trabajador trabaje tres fines de semana consecutivos.
	
	\subsection*{Naturaleza de las Variables}
	\[
	x_{i,d,s} \in \{0,1\}, \qquad y_{i,w} \in \{0,1\}
	\]
	
\end{document}
